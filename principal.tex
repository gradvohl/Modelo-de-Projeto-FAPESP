%%%%%%%%%%%%%%%%%%%%%%%%%%%%%%%%%%%%%%%%%%%%%%%%%%%%%%%%%%%%%%%%%%%%%
% In English:
%  This is a LaTeX template for São Paulo Research Foudation (FAPESP)
%  Project.
%  For information about FAPESP, visit the site 
%  http://www.fapesp.br/en
%
%  Disclaimer:
%     Although the author of this template has projects funded by 
%     FAPESP throughout his career, the research agency did not 
%     support the  construction of this code, nor it recognizes as an 
%     official template.  This template is just assistance in text 
%     formatting of projects for submission to a funding agency. 
%     Feel free to use it at your own risk.
%
%  This template targets mainly on projects written in Portuguese.
%
% In Portuguese:
%  Este é um modelo LaTeX para o projeto da Fundação de Ampario à 
%  Pesquisa do Estado de São Paulo (FAPESP).
%  Para mais informações sobre a FAPESP, visite o site 
%  http://www.fapesp.br/en
%
%  Aviso Legal:
%     Embora o autor deste modelo tenha projetos financiados pela  
%     FAPESP ao longo de sua carreira, a agência de pesquisa não   
%     apoiou a construção desse modelo, nem o reconhece como modelo
%     oficial. Este modelo é apenas uma ajuda na formatação de  
%     texto de projetos para envio a uma agência de financiamento. 
%     Sinta-se livre para usá-lo por sua conta e risco.
%
%  Este modelo tem como alvo principalmente projetos escritos em 
%  português.
%
% Author/Autor: André Leon Sampaio Gradvohl, Dr.
% Email:        gradvohl@ft.unicamp.br
% Lattes CV:    http://lattes.cnpq.br/9343261628675642
% ORCID:        https://orcid.org/0000-0002-6520-9740
% FAPESP:       https://bv.fapesp.br/en/pesquisador/102636
% 
% Last update/Última versão: 27/Oct/2019
%%%%%%%%%%%%%%%%%%%%%%%%%%%%%%%%%%%%%%%%%%%%%%%%%%%%%%%%%%%%%%%%%%%%%
%% Escolha: Portugues ou Ingles.
\documentclass[Portugues]{projetoFAPESP}
%\documentclass[Ingles]{projetoFAPESP}
%
%% Adicione o arquivo com as referências bibliográficas
\addbibresource{bibliografia.bib}
%
%% Página de título
%% Observação: As definições que aparecem a seguir comporão a
%%             página de título e a folha de rosto.
%% Define o nome da universidade onde o projeto será desenvolvido.
\universidade{Universidade Estadual de Campinas}
%
%% Define o nome da faculdade onde o projeto será desenvolvido.
\faculdade{Faculdade de Tecnologia}
%
%% Define o título do projeto.
\titulo{Análise multidimensional de sistemas para processamento online de fluxos de dados}
%
%% Define o título do projeto em inglês.
\tituloIngles{Multidimensional Analysis of Systems for Online Datastream Processing}
%
%% Define a agencia de Fomento e a abreviatura. O primeiro argumento é o 
%% nome por extenso e o segundo a abreviatura.
%% Se não houver abreviatura, deixe o segundo argumento vazio.
\agFomento{Fundação de Amparo à Pesquisa do Estado de São Paulo}{FAPESP}
%
%% Define a modalidade de projeto. 
%% Pode ser temático, regular, iniciação cientifica, mestrado, doutorado etc.
\modalidadeProjeto{Auxílio à Pesquisa Regular}
%
%% Define o autor do projeto e seu título (e.g. Dr.).
\autor{André Leon Sampaio Gradvohl}{Dr.}
%
%% Define o nome do beneficiário, se for o caso de bolsa. Esse comando é opcional.
% \beneficiario{Nome do Beneficiário}
%
%% Define a equipe executora do projeto.
%% O primeiro argumento é o nome e o segundo argumento é o título do membro (e.g. Dr.}
%% São, no máximo, 5 membros no grupo.
% \membroA{Maria da Silva}{Ph.D.}
% \membroB{Francisco José}{M.Sc.}
% \membroC{Joao dos Santos}{Bel.}
% \membroD{Xico}{Esp.}
% \membroE{Antonio}{M.D.}
% \membroF{José}{Tec.}
%
%% Define o período da vigência do Projeto.
%% Os parâmetros são dia, mês e ano. Use apenas números.
\inicioPeriodoVigencia{1}{6}{2015}
\fimPeriodoVigencia{30}{5}{2017}
%
%% Define a cidade onde o projeto será desenvolvido.
\cidade{Limeira}
%
%% Página de título
%% Observação: Os comandos a seguir não devem ser mudados em nenhuma situação.
\begin{document}
%
%% Define a numeração em romanos.
\pagenumbering{roman}
%
%% Gera a folha de título.
\geraTitulo
%
%% Gera a folha de rosto.
\folhaDeRosto
%
%% Gera a folha de rosto em inglês. 
\folhaDeRostoIngles
%
%% Escreva aqui o resumo do projeto em Português.
\begin{resumo}
  Aqui ficará o resumo em português. Sugere-se o máximo de 500 palavras.
 
  %% As palavras-chaves são opcionais, mas devem estar dentro do ambiente resumo.
  \palavraschaves{Fluxos de dados, processamento \textit{online}, \textit{benchmark}}
\end{resumo}

%
%% Escreva aqui o resumo do projeto em Inglês.
\begin{abstract}
  Here we will have the abstract in english. We suggest the maximum of 500 words.
  
  %% As palavras-chaves são opcionais, mas devem estar dentro do ambiente resumo.
  \keywords{Data Streams, Online Processing, Benchmark}
\end{abstract}
%\keywords{A, B, C}
%
%% Adicionará o sumário.
%% Mantenha os comandos \thispagestyle{empty} e \clearpage
\tableofcontents
\thispagestyle{empty}
\clearpage
%
%% Define a numeração em arábicos.
\pagenumbering{arabic}
%
%% Corpo do texto. 
%% Para melhor organização do texto, coloque cada capítulo em um arquivo separado.
%
%% Capítulo com o enunciado do problema.
\input{enunciado.tex}
%
%% Capítulo com os resultados esperados do projeto.
\input{resultadosEsperados.tex}
%
%% Capítulo com os Desafios científicos e tecnológicos e os meios e métodos para superá-los.
\input{desafios.tex}
%
%% Capítulo com o Cronograma.
\chapter{Cronograma}\label{chp:cronograma}

Quando o projeto será completado? Quais os eventos marcantes que poderão ser usados para medir o progresso do projeto e quando estará completo? Caso o projeto proposto seja parte de outro projeto maior já em andamento, estime os prazos somente para o projeto proposto.
%
%% Capítulo com Disseminação e avaliação.
\input{disseminacao.tex}
%
%% Capítulo com Outros apoios.
\input{outrosapoios.tex}
%
%% Referências bibliográficas
\printbibliography[heading=bibintoc, % Adiciona no sumário
                   title={Referências bibliográficas} % Nome do Capítulo
                  ]
%% Fim do documento
\end{document}